% LINEAR SYSTEMS AND CONTROL
% Homework 1 : Realization, controllability, observability

\documentclass[10pt,a4paper]{article}
\usepackage[latin1]{inputenc}
\usepackage{amsmath}
\usepackage{amsfonts}
\usepackage{amssymb}

\usepackage{trfsigns}
\author{Ana Huaman}
\title{Homework 1}
\begin{document}
\maketitle

% Question 1 **************************
\section{}
Sometimes there is an additional term in the output of a linear system, which takes on the form 

\[
   \begin{matrix}
   \dot{x} = Ax +Bu \\
   y = Cx + Du
   \end{matrix}
\]

What is the transfer function $G(s)$ for this type of system, where $Y(s) = G(s)U(s)$?

\subsection*{Solution}
Applying Laplace transform to both equations:
\[
  \begin{matrix}
  sX(s) = AX(s) + BU(s) \\
  Y(s) = CX(s) + DU(s)
  \end{matrix} 
\]

From the first equation we find $X(s)$:
\[ X(s) = (sI-A)^{-1}BU(s) \]

Plugging this value in the second equation:
\[ Y(s) = C(sI-A)^{-1}BU(s) + DU(s) \]
\[ Y(s) = \left ( C(sI-A)^{-1}B + D \right ) U(s) \]
\[ \dfrac{Y(s)}{U(s)} =  C(sI-A)^{-1}B + D \]
\begin{center}
\fbox{ $G(s) =  C(sI-A)^{-1}B + D$ }
\end{center}


% Question 2 **************************
\section{}
Consider the electrical network below

Using our extensive circuits knowledge, we know that

\[ y = x_{1} + C\dot{x_{2}} \]

as well as that the following relationships hold for the two loops in the network
\smallskip

Loop 1:
\[ u = Rx_{1} + L\dot{x_{1}} \]

Loop 2:
\[ u = RC\dot{x_{2}} + x_{2} \]

Find a state space realization of this system

\subsection*{Solution}
Let's order the state equations properly:

\[ \dot{x_{1}} = \frac{-R}{L}x_{1} + \frac{1}{L}u  \]
\[ \dot{x_{2}} = \frac{-1}{RC}x_{2} + \frac{1}{RC}u  \]

And replace the second equation in the expression for $y$:
\[ y = x_{1} +  C\dot{x_{2}} \]
\[ y = x_{1} +  \frac{-1}{R}x_{2} + \frac{1}{R}u \]

Now we are done!
\begin{center}
\fbox{
$
 \dot{x} 
 =
 \begin{bmatrix}
 \frac{-R}{L} & 0 \\
 0 & \frac{-1}{RC}
 \end{bmatrix}
 x
 + 
 \begin{bmatrix}
 \frac{1}{L}  \\
 \frac{1}{RC} 
 \end{bmatrix}
 u $ }
\end{center}

\begin{center}
\fbox{ 
$
 y = 
 \begin{bmatrix}
 1 &  \dfrac{-1}{R}
 \end{bmatrix}
 x
 + 
 \begin{bmatrix} 
 \dfrac{1}{R}
 \end{bmatrix}
 u
$ }
\end{center}



% Question 3 **************************
\section{}
Let
\[ y^{(4)} + 7y^{(3)} - 16\ddot{y} + 3\dot{y} + 12y = 2u - \dot{u} - 4u^{(3)} \]

\subsection*{a}
Find the Controllable Canonical Realization of this system
\subsubsection*{Solution}
Let's solve it (we have the formula nicely written in our notebooks but it is good to do it old-style once in a while).
\smallskip

Take the Laplace transform:
\[ s^{4}Y(s) + 7s^{3}Y(s) - 16s^{2}Y(s) + 3sY(s) + 12Y(s) = 2U(s) - sU(s) - 4s^{3}U(s) \]
\[ ( s^{4} + 7s^{3} - 16s^{2} + 3s + 12 )Y(s) = ( 2 - s - 4s^{3} )U(s) \]
\[ Y(s) = \dfrac{( 2 - s - 4s^{3} )}{( s^{4} + 7s^{3} - 16s^{2} + 3s + 12 )} U(s)\]
\[ Y(s) = ( 2 - s - 4s^{3} ) \cdot \dfrac{1}{( s^{4} + 7s^{3} - 16s^{2} + 3s + 12 )} U(s)\]

Let's divide the whole thing:

\[ Y(s) = G_{1}(s) \cdot G_{2}(s) \cdot U(s) \]
where:
\[ \begin{matrix}
G_{1}(s) = 2 - s - 4s^{3} \\
G_{2}(s) = \dfrac{1}{( s^{4} + 7s^{3} - 16s^{2} + 3s + 12 )} 
\end{matrix} \]

If:
\[ Z(s) = G_{2}(s)U(s) \]
\[ Y(s) = G_{1}(s)Z(s) \]

Finding $Z(s)$:
\[ Z(s) = \dfrac{1}{( s^{4} + 7s^{3} - 16s^{2} + 3s + 12 )} \cdot U(s) \]
\[ ( s^{4} + 7s^{3} - 16s^{2} + 3s + 12 )Z(s) = U(s) \]

Applying inverse Laplace Transform:
\[ z^{(4)} + 7z^{(3)} - 16\ddot{z} + 3\dot{z} + 12z = u \]

Considering: 
\begin{equation} 
\begin{matrix}
  x_{1} = z & x_{2} = \dot{z} & x_{3} = \ddot{z} & x_{4} = z^{(3)} 
  \end{matrix}
\label{eq:3_1}   
\end{equation}

we get:
\begin{center}
\fbox{
$\begin{bmatrix}
\dot{x_{1}} \\
\dot{x_{2}} \\
\dot{x_{3}} \\
\dot{x_{4}} 
\end{bmatrix}
=
\begin{bmatrix}
0 & 1 & 0 & 0 \\
0 & 0 & 1 & 0 \\
0 & 0 & 0 & 1 \\
-12 & -3 & 16 & -7
\end{bmatrix}
\begin{bmatrix}
x_{1} \\
x_{2} \\
x_{3} \\
x_{4} 
\end{bmatrix}
+
\begin{bmatrix}
0 \\
0 \\
0 \\
1 
\end{bmatrix}
u
$ }
\end{center}

And now we find $y$:

\[ Y(s) = G_{1}(s)Z(s) \]
\[ Y(s) = ( 2 - s - 4s^{3} )Z(s)\]
\[ y = 2z - \dot{z} - 4z^{(3)} \]

From Eq.(\ref{eq:3_1}), let's replace:
\[ y = 2x_{1} - x_{2} - 4x_{4} \]

And we finally get $y$ in terms of $x$:
\begin{center}
\fbox{
$y = 
\begin{bmatrix}
2 & -1  & 0 & -4
\end{bmatrix}
\begin{bmatrix}
x_{1} \\
x_{2} \\
x_{3} \\
x_{4} 
\end{bmatrix}$
}
\end{center}


\subsection*{b}
Find the Observable Canonical Realization of the system
\subsubsection*{Solution}
Now, this looks interesting! Let's check the original Differential Equation:

\[ y^{(4)} + 7y^{(3)} - 16\ddot{y} + 3\dot{y} + 12y = 2u - \dot{u} - 4u^{(3)} \]
\[ y^{(4)} = 2u - \dot{u} - 4u^{(3)} -7y^{(3)} + 16\ddot{y} - 3\dot{y} - 12y  \]

Arranging properly:
\[ y^{(4)} = - 4u^{(3)} -7y^{(3)} + 16\ddot{y} - \dot{u} - 3\dot{y} + 2u - 12y  \]

Now, let's be violent. Take integrals to both sides such that we get $y$:

\[ \iiiint y^{(4)}dt = \iiiint (- 4u^{(3)} -7y^{(3)} + 16\ddot{y} - \dot{u} - 3\dot{y} + 2u - 12y )dt \]

Assuming initial conditions zero:

\[ y = \int \left ( -4u - 7y + \int \left ( 16y + \int \left ( -u-3y + \int (2u-12y) \right  ) \right ) \right ) \]

Now:
\smallskip

Set: $x_{1} = y$
\medskip

Set: $x_{4} = \int (2u-12y) $
\smallskip

$ \dot{x_{4}} = 2u-12y \rightarrow$ \fbox{ $\dot{x_{4}} = -12x_{1} + 2u$ }
\medskip

Set: $x_{3} = \int (-u-3y + x_{4}) $
\smallskip

$ \dot{x_{3}} = -u-3y + x_{4} \rightarrow$ \fbox{ $\dot{x_{3}} = -3x_{1} + x_{4} -u$ }

Set: $x_{2} = \int (16y + x_{3}) $
\smallskip

$ \dot{x_{2}} = 16y + x_{3} \rightarrow$ \fbox{ $\dot{x_{2}} = 16x_{1} + x_{3} $ }
\medskip

Find $\dot{x_{1}}$:
$ x_{1} = y = \int (- 4u - 7y + x_{2}) \rightarrow$ \fbox{ $\dot{x_{1}} = -7x_{1} + x_{2} - 4u $ }

So, our system is now:
\begin{center}
\fbox{
$\dot{x} 
=
\begin{bmatrix}
-7 & 1 & 0 & 0 \\
16 & 0 & 1 & 0 \\
-3 & 0 & 0 & 1 \\
-12 & 0 & 0 & 0
\end{bmatrix}
x
+
\begin{bmatrix}
-4 \\
0 \\
-1 \\
2
\end{bmatrix}
u
$ }
\end{center}

and the output:

\begin{center}
\fbox{
$y = 
\begin{bmatrix}
1 & 0  & 0 & 0
\end{bmatrix}
x$ }
\end{center}


% Question 4 **************************
\section{}
A so-called unicycle model of a mobile robot states that

\[
\begin{matrix}
\dot{x_{1}} = u_{1}\cos(x_{3}) \\
\dot{x_{2}} = u_{1}\sin(x_{3}) \\
\dot{x_{3}} = u_{2}
\end{matrix}
\]

where $(x_{1}, x_{2})$ is the position of the robot, $x_{3}$ is its orientation, $u_{1}$ is its velocity and $u_{2}$ the angular velocity.

\subsection*{a}
Linearize this system around $(x_{1}, x_{2}, x_{3}) = (0,0, \dfrac{\pi}{2})$ and $(u_{1},u_{2}) = (0,0)$ in order to obtain a linear state space model.
\subsection*{Solution}
All right, from classes we know that to linearize the system, we will have to calculate an approximation such as

\[ A = \dfrac{\partial f}{\partial x} \Big |_{x_{0}, u_{0}} \]
\[ B = \dfrac{\partial f}{\partial u} \Big |_{x_{0}, u_{0}} \]
\[ C = \dfrac{\partial h}{\partial x} \Big |_{x_{0}} \]

From the equations above and from the fact that we probably want to be able to control the position as well as the orientation of the robot:

\[ f = 
\begin{bmatrix}
u_{1}\cos(x_{3}) \\
u_{1}\sin(x_{3}) \\
u_{2}
\end{bmatrix}
\]

\[ h = 
\begin{bmatrix}
x_{1}  \\
x_{2}  \\
x_{3}
\end{bmatrix}
\]

Using the equations above, let's do some derivations:

\[ A = 
\begin{bmatrix}
0 & 0 & -u_{1}\sin(x_{3}) \\
0 & 0 & u_{1}\cos(x_{3}) \\
0 & 0 & 0
\end{bmatrix}
\]

\[ B = 
\begin{bmatrix}
\cos(x_{3}) & 0 \\
\sin(x_{3}) & 0 \\
0 & 1
\end{bmatrix}
\]

\[ C =
\begin{bmatrix}
1 & 0 & 0 \\
0 & 1 & 0 \\
0 & 0 & 1
\end{bmatrix}
\]

Evaluating it around $(x_{1}, x_{2}, x_{3}) = (0,0, \dfrac{\pi}{2})$ and $(u_{1},u_{2}) = (0,0)$:

\[ A = 
\begin{bmatrix}
0 & 0 & 0 \\
0 & 0 & 0 \\
0 & 0 & 0
\end{bmatrix}
\]

\[ B = 
\begin{bmatrix}
0 & 0 \\
1 & 0 \\
0 & 1
\end{bmatrix}
\]

\[ C =
\begin{bmatrix}
1 & 0 & 0 \\
0 & 1 & 0 \\
0 & 0 & 1
\end{bmatrix}
\]

\subsection*{b}
Without worrrying too much about the  technical definition, a system is controllable if it is possible to drive the system between any two states (by picking a suitably clever input signal). Is the system in {\bf 4.a} controllable?
\subsection*{Solution}
The unicycle (as it is presented in this question) is controllable. To achieve any combination of position $+$ orientation, the control input should be in its simplest form a combination of three steps: rotation + translation + rotation.
\medskip
 
Let's say that we want to get to position $(x_{G},y_{G})$ with orientation $\theta_{G}$. What we would do would be:

\begin{itemize}
\item{Rotate our robot so our new orientation is $\tan^{-1}(y_{G}/x_{G})$}
\item{Translate forward until reaching the position  $(x_{G},y_{G})$}
\item{Rotate in our place until we reach the desired orientation $\theta_{G}$}
\end{itemize}

If we want to get fancy we might prove this by Non-Linear Controllability theory, but I don't know that much yet (only read it out of curiosity because of this problem), in any case, I would like to write it down so I will remember it in the future:

Let's check the non-linear controllability matrix: 
\[
\begin{bmatrix}
\cos \theta & 0 & \sin \theta \\
\sin \theta & 0 & -\cos \theta \\
0 & 1 & 0
\end{bmatrix}
\]

The two first columns come from $g(x)$, which in turns comes from $\dot{x} = g_{1}(x)u_{1} + g_{2}(x)u_{2}$. The third column comes from the Lie bracket:

\[ [g_{1}, g_{2}] = \dfrac{\partial g_{2}}{\partial x} g_{1} - \dfrac{\partial g_{1}}{\partial x} g_{2} = 
\begin{bmatrix}
\sin \theta \\
- \cos \theta \\
0
\end{bmatrix}
\] 

With some quick math we can verify that the rank of this matrix is $r = n = 3$ ( $n$ the number of states). Hence, the system is controllable.

% Question 5 **************************
\section{}
Control systems can be found everywhere. Look around you and list 5 everyday devices that require control for their operation (By control, I mean they have to measure something and then change their dynamics, i.e. what they are doing, accordingly)
\subsection*{Solution}
\begin{itemize}
\item{ Outdoor lighting: Most of us have this installed at home. The lights turn on (change) when movement is detected nearby (input)}
\item{ Automatic Sprinkler : Having as an input a sensorial signal that indicates that the environment is dry, the system changes and turn on the water sprinkler when needed.}
\item{ Automatic doors: You see them everyday when you go to Publix. When the presence of a person is detected nearby, the door opens automatically, then closes when you go away. }
\item{ Your car: The input would come from the wheel (for orientation) and from the gas pedal for speed (acceleration). The system changes accordingly.}
\item{ An oven: You manipulate the handle to indicate the temperature and the oven changes the temperature based on the new handle position}
\item{ Your shower: You have hot or cold water by rotating the handle. Depending on how much you rotate the tap, the water temperature will be different.}
\end{itemize}
\end{document}